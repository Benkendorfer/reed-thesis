\documentclass[../thesis.tex]{subfiles}

\providecommand{\zcut}{z_\mathrm{{cut}}}
\providecommand{\LIPS}{\mathrm{LIPS}}
\providecommand{\cusp}{\mathrm{cusp}}
\providecommand{\mMDT}{\mathrm{mMDT}}
\providecommand{\Li}{\mathrm{Li}}

\providecommand{\arctanh}{\mathrm{arctanh}}

\providecommand{\cM}{\mathcal{M}}
\providecommand{\cL}{\mathcal{L}}
\providecommand{\cO}{\mathcal{O}}


\setlength{\parskip}{0pt}
%%
%% End Preamble
%%
%% The fun begins:
\begin{document}
	\textit{Forsan et haec olim meminisse iuvabit.}

	\,\\

	\noindent Endings are difficult for me. Sometimes, looking back on all the good times of the past, it is challenging to let go and accept that the future will be similarly bright. Even letting go of the many struggles is hard, knowing that all I have managed to overcome so far will be followed by yet more trials to come. These feelings are coming on especially strongly now, as I sit down to write and reflect on the last four years.

	More than anything, the people here have made my time at Reed special. The list of people to whom I am grateful is far too long to properly enumerate. Everybody I have interacted with here has left their mark on me in one way or another. Thus, while I must acknowledge incompletely, know that there are many more not mentioned.

	As clich\'e as it may be, I am enormously grateful to my parents for the support you have provided me. These last four years have been immensely trying back in North Carolina, but I am so thankful that you have still found the ability to keep me going from thousands of miles away. Here's hoping that the next four years are a little easier on us!

	I have also been blessed by a fantastic roster of professors. There are a few who stand out. Joel, I hope someday to be able to talk about physics as clearly as you do. The hours I have spent in your office have carried me through many classes, your sense of humor has always been a source of joy, and your `opinions' on various topics have been an immense help throughout my four years here. Lucas, thank you for first teaching me about the joy of research. You have helped me find a very rewarding path. Andrew, your passion for particle physics has helped light in me a fire that I suspect will burn for many years to come. More than any other person, through your feedback, advice, and connections, you have single-handedly changed my trajectory for the better. Thank you for being both an excellent thesis advisor and an excellent mentor.

	Finally, maybe the real thesis is the friends we made along the way. Pax and Maggie, you left a deep mark on my first years here. Though we may not talk much anymore, I won't forget the (several!) times we almost froze to death in the Oregon mountains. Jagannath, even three years later, I still miss our late night conversations in the McKinnley kitchen (and the smell of chai spilled on the burner).

	Saba, Dennis, Amelia, and Maham, there is nothing like working on a problem set in P123 until midnight, then following it with truck stop burgers at Jubitz. We had our differences in the end, but I'm still so glad to have known y'all.

	Thomas, you have always managed to make me laugh. I don't know who else could have simultaneously dragged me on a trans-continental train trip, convinced me to take an economics class on monetary and fiscal policy, and turned me into a radical anti-suburbanite. Even across the Atlantic, you're the best Trains Supervisor any (S)RO could hope to have.

	Val, Kaitlyn, and the rest of the reactor climbing crew: our weekly Nong's and climbing outings form some of my happiest memories from the past years. I'm so sad that the pandemic got in the way of our fledgling tradition just as we were starting to get strong! But it was fun while it lasted.

	Sam, it has been a real treat to know you as you come into your own. Seeing that people like you are on the ball gives me hope that maybe, just maybe, our planet will be alright. M, you are an absolute saint --- keep on being a cool cat. I hope you both have a wonderful senior year.

	Orion and Hima, your constant energy is continually astounding, and your positive attitudes (even in the face of hardship) have been a great source of joy. I hope you can forgive me for not moving to the Bay! Good luck with your last two years, and please, for the love of God, don't buy any more expensive bikes.

	To the Boys --- Amelia, Hannah, Hart, Maya, and Clara --- y'all are some of the nicest people I have met. Knowing you has made this COVID year a little more bearable. Hart, I still can't believe we drove not once, but twice, across the country because of this pandemic. It's wild that we could decide to undertake something like that and then hit the road a couple days later. Amelia and Hannah, props to you for getting me to start running regularly. Our thrice-weekly outings have given me something to look forward to every week, even through all the trials of senior year. While it's unfortunate that our two years of separation mean the end of this, I wish you both the best of luck as you continue your own journeys through this school. I'll hold out hope that our paths will cross again someday.

	Last, but certainly not least: Gwen. I'm so, so grateful that of all the open seats in 201 lab, you chose to sit next to me. You helped me heal when I was injured; you calmed me when I was stressed; and you comforted me when I was sad. You made me laugh when we were happy; you pushed me when I had to be pushed; and you helped convince me that it's ok (healthy, even) to take a break now and then. You're the best bean, and I'm so glad to know you.

	There is so much more to be said, but at some point, I guess we have to accept the end. It's time to get to work.
\end{document}
