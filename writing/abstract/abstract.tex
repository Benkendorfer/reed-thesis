\documentclass[../thesis.tex]{subfiles}

\providecommand{\zcut}{z_\mathrm{{cut}}}
\providecommand{\LIPS}{\mathrm{LIPS}}
\providecommand{\cusp}{\mathrm{cusp}}
\providecommand{\mMDT}{\mathrm{mMDT}}
\providecommand{\Li}{\mathrm{Li}}

\providecommand{\arctanh}{\mathrm{arctanh}}

\providecommand{\cM}{\mathcal{M}}
\providecommand{\cL}{\mathcal{L}}
\providecommand{\cO}{\mathcal{O}}


\setlength{\parskip}{0pt}
%%
%% End Preamble
%%
%% The fun begins:
\begin{document}
	Imposing external scales on a perturbative calculation within the framework of quantum chromodynamics (QCD) introduces corrections that take the form of logarithms of ratios of energy scales. If the scales are very different, these logarithms can grow large. As a result, the precision calculation of observables in QCD near phase space boundaries (such as low-energy limits) requires careful accounting of such logarithms through the process of resummation. 

	At its heart, this thesis explores the techniques of resummation. We do so through the example of an all-orders calculation of heavy hemisphere mass in $e^+ e^- \to \text{jets}$ events under the influence of jet grooming using the modified mass drop tagger (mMDT) algorithm. We will work in the limit $\rho \sim \zcut \ll 1$, where the heavy hemisphere mass $\rho$ is small but of the same order as the mMDT grooming cutoff $\zcut$.

	In Chapters \ref{chap:intro} and \ref{chap:technical}, we introduce the physical background necessary to place the calculation in context. In Chapter \ref{chap:leading order}, we perform a fixed-order calculation of the distribution in order to develop familiarity with important mathematical techniques like dimensional regularization. Finally, in Chapters \ref{chap:factorization} and \ref{chap:all orders}, we develop the framework for all-orders resummation of groomed heavy hemisphere mass. At the very end, we demonstrate how one would use the framework to calculate the distribution to next-to-leading-logarithmic (NLL) accuracy.
\end{document}
