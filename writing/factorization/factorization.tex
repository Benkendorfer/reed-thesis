% This is the Reed College LaTeX thesis template. Most of the work 
% for the document class was done by Sam Noble (SN), as well as this
% template. Later comments etc. by Ben Salzberg (BTS). Additional
% restructuring and APA support by Jess Youngberg (JY).
% Your comments and suggestions are more than welcome; please email
% them to cus@reed.edu
%
% See http://web.reed.edu/cis/help/latex.html for help. There are a 
% great bunch of help pages there, with notes on
% getting started, bibtex, etc. Go there and read it if you're not
% already familiar with LaTeX.
%
% Any line that starts with a percent symbol is a comment. 
% They won't show up in the document, and are useful for notes 
% to yourself and explaining commands. 
% Commenting also removes a line from the document; 
% very handy for troubleshooting problems. -BTS

% As far as I know, this follows the requirements laid out in 
% the 2002-2003 Senior Handbook. Ask a librarian to check the 
% document before binding. -SN

%%
%% Preamble
%%
% \documentclass{<something>} must begin each LaTeX document
\documentclass[12pt,twoside,class=../reedthesis, crop=false]{standalone}
% Packages are extensions to the basic LaTeX functions. Whatever you
% want to typeset, there is probably a package out there for it.
% Chemistry (chemtex), screenplays, you name it.
% Check out CTAN to see: http://www.ctan.org/
%%
\usepackage{graphicx,latexsym} 
\usepackage{amssymb,amsthm,amsmath}
\usepackage{longtable,booktabs,setspace} 
\usepackage{chemarr} %% Useful for one reaction arrow, useless if you're not a chem major
\usepackage[hyphens]{url}
\usepackage{rotating}
\usepackage{hyperref}

\usepackage{physics}
\usepackage{siunitx}
\usepackage{xcolor}
% \usepackage{standalone}
% \usepackage{natbib}
% Comment out the natbib line above and uncomment the following two lines to use the new 
% biblatex-chicago style, for Chicago A. Also make some changes at the end where the 
% bibliography is included. 
%\usepackage{biblatex-chicago}
%\bibliography{thesis}

% \usepackage{times} % other fonts are available like times, bookman, charter, palatino

\newcommand{\zcut}{\mathrm{z_{cut}}}


\setlength{\parskip}{0pt}
%%
%% End Preamble
%%
%% The fun begins:
\begin{document}
	The first step on the path to an all-orders calculation is to derive a factorization formula for the heavy hemisphere mass cross section. The basic process for doing so is laid out in technical detail in Ref.~\cite{becher_introduction_2015-1}, and an example of a similar flavor to our calculation is provided by Frye et al.\ in Ref.~\cite{frye_factorization_2016}.\footnote{Indeed, the calculation of Frye et al.\ is a more general factorization of mass-like variables in groomed jets. Setting $\alpha = 2, \beta = 0$ for their two-point energy correlation function $e_2^{(\alpha)}$ under soft drop grooming with angular exponent $\beta$ yields the mMDT-groomed jet mass $\rho$. Their factorization is valid in the limit $\rho \ll \zcut \ll 1$, whereas we are interested in the limit $\rho \sim \zcut \ll 1$.} There are two primary steps in developing a factorization formula:
	\begin{enumerate}
		\item \textbf{Power counting}: this involves determining the possible radiative modes of an event and their dominant momentum scales. The term `power counting' refers to the fact that for some momentum scale $\lambda$, different radiative modes have momenta that scale as different powers of $\lambda$.

		\item \textbf{Factorization and refactorization}: Once the different radiative modes and energy scales are identified, we can use the framework of SCET to split the cross section into a convolution of terms describing different radiative modes. These terms themselves must then be split (refactored) into convolutions of terms, each of which depends, to leading order, only on a single energy scale.
	\end{enumerate}

\section{Setup}
	Throughout the following discussion, with $n^\mu$ the jet direction and $\bar n^\mu$ the direction opposite the jet, we will describe momenta in light-cone coordinates
	\begin{equation}
		p^\mu = \qty(p^-, p^+, p_\perp)
	\end{equation}
	with
	\begin{align}
		p^- &= \bar n \cdot p & p^+ &= n \cdot p
	\end{align}
	and $p_\perp$ the components of momentum transverse to $n$.

	Recall that the hemisphere mass is defined to be
	\begin{equation}
		\rho = \frac{1}{E_J^2} \sum_{i<j} 2p_i \cdot p_j
	\end{equation}
	with $E_J$ the jet energy and the sum ranging over all pairs of particles in the jet. Expanding out the dot product, we have
	\begin{equation}
		\rho = \frac{2}{E_J^2} \sum_{i<j} \qty(E_i E_j - \vb{p}_i \cdot \vb{p}_j) = \frac{2}{E_J^2} \sum_{i<j} E_i E_j \qty(1 - \cos\theta_{ij}) = \sum_{i<j}2z_i z_j \qty(1 - \cos\theta_{ij}).
	\end{equation}
	Here, $z_i$ and $z_j$ are the relative energy fractions of each particle and $\theta_{ij}$ is the angle between particles $i$ and $j$.
	
	In an $e^+ e^- \to \text{jets}$ event, there are two types of emission: resolved and unresolved. The essential difference is that a resolved emission is one which manifests itself as a jet at a particular scale of observation, while an unresolved emission does not. The presence of unresolved emissions can, however, perturb observable values of a resolved emission. {\color{red}\textbf{[TODO: check that this is a reasonable description]}} 

	Suppose now that we have applied an mMDT groomer with energy fraction cutoff $\zcut$. Then every \textit{resolved} emission must satisfy
	\begin{equation}
		z_i > \zcut,
	\end{equation}
	while other emissions with $z_i < \zcut$ can only pass the groomer if they are at a sufficiently small angle to a resolved emission.

\section{Power counting}
\section{Factorization}

	\ifstandalone
	\bibliographystyle{../bsts/JHEP} 
	\bibliography{../jet_substructure}
	\fi
\end{document}
