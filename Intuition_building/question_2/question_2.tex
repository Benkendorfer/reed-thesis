\documentclass[11pt,twoside,reqno]{amsart}
\usepackage{amssymb, amsmath, enumerate, palatino, hyperref}
\usepackage[normalem]{ulem}
%\usepackage{fullpage}
\usepackage[margin=1in]{geometry}
\usepackage[T1]{fontenc}
\renewcommand{\labelitemi}{\guillemotright}
\usepackage{mathrsfs}
%--------------------------------
% Kees' imports
%--------------------------------
\usepackage{physics}
\usepackage{tensor}
\usepackage{mathtools}
\usepackage{xcolor}
\usepackage{siunitx}
\usepackage{empheq}
\usepackage{tensor}


\theoremstyle{plain}
\newtheorem{prop}{Proposition}%[section]
\newtheorem{lemma}[prop]{Lemma}
\newtheorem{thm}[prop]{Theorem}
\newtheorem{obs}[prop]{Observation}
\newtheorem{app}[prop]{Application}
\newtheorem*{MainThm}{Main Theorem}
\newtheorem{cor}[prop]{Corollary}
\newtheorem{conj}[prop]{Conjecture}
\theoremstyle{remark}
\newtheorem{rmk}[prop]{Remark}
\theoremstyle{definition}
\newtheorem{prob}{Problem}
\newtheorem{bonus}[prop]{Bonus Problem}
\theoremstyle{remark}
\newtheorem{exc}{Exercise}
\newtheorem*{soln}{Solution}
\theoremstyle{definition}
\newtheorem{ex}[prop]{Example}
\theoremstyle{definition}
\newtheorem{defn}[prop]{Definition}

\newcommand{\RR}{\mathbb{R}}
\newcommand{\ZZ}{\mathbb{Z}}
\newcommand{\CC}{\mathbb{C}}
\newcommand{\NN}{\mathbb{N}}
\newcommand{\QQ}{\mathbb{Q}}

\newcommand{\Aut}{\operatorname{Aut}}

\newcommand{\defeq}{:=}

\renewcommand{\Re}{\operatorname{Re}}

%--------------------------------
% Kees' commands
%--------------------------------
\makeatletter
\newcommand{\subalign}[1]{%
  \vcenter{%
    \Let@ \restore@math@cr \default@tag
    \baselineskip\fontdimen10 \scriptfont\tw@
    \advance\baselineskip\fontdimen12 \scriptfont\tw@
    \lineskip\thr@@\fontdimen8 \scriptfont\thr@@
    \lineskiplimit\lineskip
    \ialign{\hfil$\m@th\scriptstyle##$&$\m@th\scriptstyle{}##$\hfil\crcr
      #1\crcr
    }%
  }%
}
\makeatother

\newcommand{\allspace}{{\substack{\text{all}\\\text{space}}}}

\newcommand{\DD}{\mathbb{D}}
\renewcommand{\AA}{\mathbb{A}}
\newcommand{\VV}{\mathbb{V}}
\newcommand{\LL}{\mathbb{L}}
\newcommand{\BB}{\mathbb{B}}
\renewcommand{\SS}{\mathbb{S}}
\newcommand{\HH}{\mathbb{H}}

\renewcommand{\l}{\ell}

\newcommand{\PB}{\mathrm{PB}}

\newcommand{\cL}{\mathcal{L}}
\newcommand{\cE}{\mathcal{E}}
\newcommand{\cH}{\mathcal{H}}
\newcommand{\cC}{\mathcal{C}}
\newcommand{\cA}{\mathcal{A}}
\newcommand{\cI}{\mathcal{I}}

\newcommand{\sgn}{\mathrm{sgn}}

\def\beq#1\eeq{\begin{equation}#1\end{equation}}
\def\bal#1\eal{\begin{align}#1\end{align}}

\title{Limiting Eq.\ 2.7 with $x_3 \ll 1$}
\author{Kees Benkendorfer}
\date{5 October 2020}

\begin{document}
\maketitle

We want to take the $x_3 \ll 1$ limit of the integral from Eq.\ 2.7 of 2006.14680:
\begin{equation}
	I = \int_0^1 dx_1 \int_0^1 dx_2 \, \Theta(x_1 + x_2 - 1) \frac{x_1^2 + x_2^2}{(1 - x_1)(1 - x_2)} \, \delta \qty(\rho - \frac{4(1 - \max\qty{x_i})}{(2 - \max\qty{x_i})^2}) \Theta\qty(\frac{\min\qty{x_i}}{2 - \max\qty{x_i}} - z),
\end{equation}
where $x_1, x_2, x_3$ are phase space variables satisfying
\begin{equation}
	x_1 + x_2 + x_3 = 2.
\end{equation}
As a first step, we can change variables from $x_2$ to $x_3$:
\begin{align}
	x_3 &= 2 - x_1 - x_2 & dx_2 &= -dx_3.
\end{align}
Then, after partitioning unity with the definition $\Theta_{ijk} \equiv \Theta(x_i - x_j)\Theta(x_j - x_k)$, the integral becomes
\begin{equation}
\begin{aligned}
	I &= \int_0^1 dx_3 \int_{1 - x_3}^{2 - x_3} dx_1 \, \Theta(1 - x_3) \frac{x_1^2 + (2 - x_1 - x_3)^2}{(1 - x_1)(x_1 + x_3 - 1)} \, \delta \qty(\rho - \frac{4(1 - \max\qty{x_i})}{(2 - \max\qty{x_i})^2}) \\
	&\qquad\qquad\qquad\qquad \times \Theta\qty(\frac{\min\qty{x_i}}{2 - \max\qty{x_i}} - z) \Big[ \Theta_{123} + \Theta_{132} + \Theta_{213} + \Theta_{231} + \Theta_{312} + \Theta_{321} \Big].
\end{aligned}
\end{equation}
Taking a look at the first integral, we have
\begin{equation}
\begin{aligned}
	I_1 &= \int_0^1 dx_3 \int_{1 - x_3}^{2 - x_3} dx_1 \Theta(1 - x_3) \frac{x_1^2 + (2 - x_1 - x_3)^2}{(1 - x_1)(x_1 + x_3 - 1)} \, \delta \qty(\rho - \frac{4(1 - x_1)}{(2 - x_1)^2}) \\
		&\qquad\qquad\qquad\qquad \times \Theta\qty(\frac{x_3}{2 - x_1} - z) \Theta(2x_1 + x_3 - 2)\Theta(2 - 2x_1 - x_3).
\end{aligned}
\end{equation}
We can simplify the Dirac delta by considering its argument to be a function of $x_1$
\begin{equation}
	f(x_1) = \rho - \frac{4(1 - x_1)}{(2 - x_1)^2}.
\end{equation}
Then its roots are
\begin{equation}
	r_1, r_2 = 2 + \frac{2\qty(-1 \pm \sqrt{1 - \rho}\,)}{\rho},
\end{equation}
so
\begin{equation}
	\delta\qty(\rho - \frac{4(1 - x_1)}{(2 - x_1)^2}) = \frac{\delta(x_1 - r_1)}{\abs{f'(r_1)}} + \frac{\delta(x_1 - r_2)}{\abs{f'(r_2)}}.
\end{equation}
Only the first root will contribute in this case (since the other is negative for $0 < \rho < 1$), so
\begin{align}
	I_1 &= \int_0^1 dx_3 \int_{1 - x_3}^{2 - x_3} dx_1 \Theta(1 - x_3) \frac{x_1^2 + (2 - x_1 - x_3)^2}{(1 - x_1)(x_1 + x_3 - 1)} \, \frac{\delta(x_1 - r_1)}{\abs{f'(r_1)}} \\
		\nonumber&\qquad\qquad\qquad\qquad \times \Theta\qty(\frac{x_3}{2 - x_1} - z) \Theta(2x_1 + x_3 - 2)\Theta(2 - 2x_1 - x_3) \\
	&= \int_0^1 dx_3\, \Theta(1 - x_3) \frac{r_1^2 + (2 - r_1 - x_3)^2}{(1 - r_1)(r_1 + x_3 - 1)} \, \frac{1}{\abs{f'(r_1)}} \\
		\nonumber&\qquad\qquad\qquad\qquad \times \Theta\qty(\frac{x_3}{2 - r_1} - z) \Theta(2r_1 + x_3 - 2)\Theta(2 - 2r_1 - x_3) \\
	&\stackrel{?}{\approx} \int_0^\infty dx_3\, \Theta(1 - x_3) \frac{4 - 4r_1 + 2r_1^2 + 2r_x x_3 - 4x_3}{(1 - r_1)(r_1 + x_3 - 1)} \, \frac{1}{\abs{f'(r_1)}} \\
		\nonumber&\qquad\qquad\qquad\qquad \times \Theta\qty(\frac{x_3}{2 - r_1} - z) \Theta(2r_1 + x_3 - 2)\Theta(2 - 2r_1 - x_3).
\end{align}
Here's where I'm suck: in the limit $x_3 \ll 1$, we might take the upper bound on the integral to be $\infty$ (as you suggest), which would simplify the integration (although\dots\ how then to handle the upper bounds on $x_3$ imposed by the Heaviside functions?). But how can we make the approximation play nice with, for example, the requirement
\begin{equation}
	\Theta(2r_1 + x_3 - 2) \implies x_3 > 2(1 - r_1)?
\end{equation}
These imposed lower bounds seem to prevent us from satisfying the limit $x_3 \to 0$. Should we treat them as a bound on $\rho$ instead (e.g.\ by saying $\Theta(2r_1 + x_3 - 2) \approx \Theta(2r_1 - 2$)? Then it seems like we're potentially losing an important part of the integral.

Do you have any pointers on this?

\vspace{1cm}

\hrulefill

\vspace{1cm}

Ok, since we are setting $x_3 \ll 1$, we know $1 - x_1 \ll 1$, so let's set
\begin{align}
	x_3 &\to \lambda x_3 & x_1 \to 1 - \lambda(1 - x_1),
\end{align}
then drop terms quadratic and higher in $\lambda$. Since the delta function and Heaviside functions are linear in $x_1$ and $x_3$
	
\end{document}