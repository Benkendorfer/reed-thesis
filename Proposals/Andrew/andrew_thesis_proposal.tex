\documentclass[11pt,twoside,reqno]{amsart}
\usepackage{amssymb, amsmath, enumerate, palatino, hyperref}
\usepackage[normalem]{ulem}
%\usepackage{fullpage}
\usepackage[margin=1in]{geometry}
\usepackage[T1]{fontenc}
\renewcommand{\labelitemi}{\guillemotright}
\usepackage{mathrsfs}
%--------------------------------
% Kees' imports
%--------------------------------
\usepackage{physics}
\usepackage{tensor}
\usepackage{mathtools}
\usepackage{xcolor}
\usepackage{siunitx}
\usepackage{empheq}
\usepackage{tensor}


\theoremstyle{plain}
\newtheorem{prop}{Proposition}%[section]
\newtheorem{lemma}[prop]{Lemma}
\newtheorem{thm}[prop]{Theorem}
\newtheorem{obs}[prop]{Observation}
\newtheorem{app}[prop]{Application}
\newtheorem*{MainThm}{Main Theorem}
\newtheorem{cor}[prop]{Corollary}
\newtheorem{conj}[prop]{Conjecture}
\theoremstyle{remark}
\newtheorem{rmk}[prop]{Remark}
\theoremstyle{definition}
\newtheorem{prob}{Problem}
\newtheorem{bonus}[prop]{Bonus Problem}
\theoremstyle{remark}
\newtheorem{exc}{Exercise}
\newtheorem*{soln}{Solution}
\theoremstyle{definition}
\newtheorem{ex}[prop]{Example}
\theoremstyle{definition}
\newtheorem{defn}[prop]{Definition}

\newcommand{\RR}{\mathbb{R}}
\newcommand{\ZZ}{\mathbb{Z}}
\newcommand{\CC}{\mathbb{C}}
\newcommand{\NN}{\mathbb{N}}
\newcommand{\QQ}{\mathbb{Q}}

\newcommand{\Aut}{\operatorname{Aut}}

\newcommand{\defeq}{:=}

\renewcommand{\Re}{\operatorname{Re}}

%--------------------------------
% Kees' commands
%--------------------------------
\makeatletter
\newcommand{\subalign}[1]{%
  \vcenter{%
    \Let@ \restore@math@cr \default@tag
    \baselineskip\fontdimen10 \scriptfont\tw@
    \advance\baselineskip\fontdimen12 \scriptfont\tw@
    \lineskip\thr@@\fontdimen8 \scriptfont\thr@@
    \lineskiplimit\lineskip
    \ialign{\hfil$\m@th\scriptstyle##$&$\m@th\scriptstyle{}##$\hfil\crcr
      #1\crcr
    }%
  }%
}
\makeatother

\newcommand{\allspace}{{\substack{\text{all}\\\text{space}}}}

\newcommand{\DD}{\mathbb{D}}
\renewcommand{\AA}{\mathbb{A}}
\newcommand{\VV}{\mathbb{V}}
\newcommand{\LL}{\mathbb{L}}
\newcommand{\BB}{\mathbb{B}}
\renewcommand{\SS}{\mathbb{S}}
\newcommand{\HH}{\mathbb{H}}

\renewcommand{\l}{\ell}

\newcommand{\PB}{\mathrm{PB}}

\newcommand{\cL}{\mathcal{L}}
\newcommand{\cE}{\mathcal{E}}
\newcommand{\cH}{\mathcal{H}}
\newcommand{\cC}{\mathcal{C}}
\newcommand{\cA}{\mathcal{A}}
\newcommand{\cI}{\mathcal{I}}

\newcommand{\sgn}{\mathrm{sgn}}

\def\beq#1\eeq{\begin{equation}#1\end{equation}}
\def\bal#1\eal{\begin{align}#1\end{align}}

\title{Thesis proposal: precision calculations of groomed jet mass}
\author{Kees Benkendorfer}
\date{11 September 2020}

\begin{document}
\maketitle

\noindent\textbf{Proposed Advisor:} Andrew Larkoski \\
\noindent\textbf{Preference:} 1 \\

In the collision of two high-energy particles (e.g.\ electron-positron or proton-proton), the emission of a quark or a gluon leads to the production of a collimated, structured spray of particles called a jet. This is a direct result of the self-coupling of the strong force, described by the theory of quantum chromodynamics (QCD) \cite{larkoski_elementary_2019-1}. Since these jets are such a `purely' QCD phenomenon, they provide an excellent laboratory in which to study the theory directly. One would like, for example, to use jet substructure to measure the strong coupling $\alpha_s$, a fundamental quantity of QCD (if not \textit{the} fundamental quantity).

Yet in order to measure quantities using jet substructure, we must have a clear theoretical understanding against which to compare experimental data. This is difficult due to the nature of QCD. A precise theoretical treatment of jets must contend with a breakdown of perturbation techniques (since $\alpha_s \sim 1$) and significant nonlinear corrections due to hadronization effects (since particles that have a charge under the strong force become `confined' together). Moreover, in an experimental context, one must account for radiation introduced to a jet from other events in the collider.

We must, therefore, study jet observables that are robust to these problems. One such observable is the `groomed jet mass' --- the total mass/energy of all the particles in the jet, corrected to remove the low-energy and very off-axis particles typical of background radiation. This observable was recently calculated for the first time to high enough precision ($N^2LO + N^3LL$, for the \textit{cognoscenti}) \cite{kardos_groomed_2020,kardos_two-_2020} for comparison to data to be feasible\footnote{This is very timely, as the ATLAS collaboration recently produced a new measurement of groomed jet mass \cite{atlas_collaboration_measurement_2020-2}}. However, this calculation is only valid at low and high values of groomed jet mass (relative to the center-of-mass energy of the collided particles); intermediate values fall outside the regions of validity. Yet interesting physics occurs here, as evidenced by the significant cusp in Fig.\ 4 of Ref.\ \cite{kardos_groomed_2020}, and so understanding this region is crucial to the ultimate goal of computing $\alpha_s$.

For this thesis, I would perform calculations to understand the physics of this intermediate-mass region, with the goal of pushing the calculation to the precision achieved for the tails of the distribution. If this proceeded swiftly, I would then attempt a (very rough) determination of $\alpha_s$. This project would provide an excellent opportunity to learn about QCD, jet phenomenology, and the techniques of modern theoretical high-energy physics.

\bibliographystyle{unsrt}
\bibliography{jet_substructure}

\end{document}